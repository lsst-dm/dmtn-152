\documentclass[DM,authoryear,toc]{lsstdoc}
% lsstdoc documentation: https://lsst-texmf.lsst.io/lsstdoc.html
\input{meta}

% Package imports go here.

% Local commands go here.

%If you want glossaries
%\input{aglossary.tex}
%\makeglossaries

\title{Rubin Algorithms Workshop - Scientific Summary}

% Optional subtitle
% \setDocSubtitle{A subtitle}

\author{%
Leanne Guy and  \v{Z}eljko Ivezi\'{c}
}

\setDocRef{DMTN-152}
\setDocUpstreamLocation{\url{https://github.com/lsst-dm/dmtn-152}}

\date{\vcsDate}

% Optional: name of the document's curator
% \setDocCurator{The Curator of this Document}

\setDocAbstract{%
In March 2020, Rubin Observatory hosted a workshop on the topic of image process- ing algorithms for the Legacy Survey of Space and Time (LSST), (\cite{RAW2020}).  This technote is a summary of that workshop
}

% Change history defined here.
% Order: oldest first.
% Fields: VERSION, DATE, DESCRIPTION, OWNER NAME.
% See LPM-51 for version number policy.
\setDocChangeRecord{%
  \addtohist{1}{2020-05-26}{First draft.}{Leanne Guy}
}


\begin{document}

% Create the title page.
\maketitle
% Frequently for a technote we do not want a title page  uncomment this to remove the title page and changelog.
% use \mkshorttitle to remove the extra pages

% ADD CONTENT HERE
% You can also use the \input command to include several content files.

\section{Introduction}
The Rubin science pipelines are described in \citeds{LDM-148} and the resultant data products in \citeds{LSE-163}.

% The summary of the topics presented 
\section{Summary of topics discussed}
The following provides a summary of the presentations and discussions held at the Rubin Algorithms Workshop.
All talks and presentations are linked from the workshop web page (\cite{RAW2020}).

% Introduction to the LSST Science Pipelines. 
% \subsection{Introduction to the science pipelines}

\subsection{Instrument Signature Removal}
 \textbf{ \textit{Speaker:} Robert Lupton (Rubin) } 

% Discussion from Mike's talk should be woven in to the PSF summary
\subsection{PSF estimation}
 \textbf{ \textit{Speakers:} Josh Meyers (Rubin), Mike Jarvis (Invited) } 

\subsection{Deblending}
 \textbf{ \textit{Speaker:} Fred Moolekamp (Rubin) } 

\subsection{Building and Using Coadds}
 \textbf{ \textit{Speaker:} Jim Bosch (Rubin) } 

\subsection{Background Estimation}
 \textbf{ \textit{Speaker:} Yusra AlSyyaad(Rubin) } 

% Discussion from Gary's talk should be woven in to the calibration summaries
\subsection{Photometric Calibration}
 \textbf{ \textit{Speaker:} Eli Rykoff (Rubin),  Gary Bernstein  (Invited)  } 

\subsection{Astrometric Calibration}
 \textbf{ \textit{Speaker:} Jim  Bosch (Rubin), Gary Bernstein (Invited)  } 

% Discussion from Gene's talk should be woven in to the Difference Imaging summary
\subsection{Difference Imaging}
 \textbf{ \textit{Speakers:} Eric Bellm (Rubin), Gene Magnier (Invited)  } 

\subsection{DCR in Templates}
 \textbf{ \textit{Speakers:} Ian Sullivan (Rubin) } 
  
 % Discussion from Konrad's talk should be woven in to the Galaxy Photometry summary
 \subsection{Galaxy Photometry}
 \textbf{ \textit{Speakers:} Jim  Bosch (Rubin), Dan Taranu (Rubin), Konrad Kuijken (Invited)  } 

 % Discussion from Erin's talk should be woven in to Shape Measurements summary
  \subsection{Shape  Measurements }
   \textbf{ \textit{Speakers:} Jim  Bosch (Rubin),  Erin  Sheldon  (Invited)  } 
  
  \subsection{Stellar Crowded Fields}
 \textbf{ \textit{Speakers:} Colin Slater  (Rubin) } 

\section{Conclusions}
\appendix
% Include all the relevant bib files.
% https://lsst-texmf.lsst.io/lsstdoc.html#bibliographies
\section{References} \label{sec:bib}
\bibliography{local,lsst,lsst-dm,refs_ads,refs,books}

% Make sure lsst-texmf/bin/generateAcronyms.py is in your path
\section{Acronyms} \label{sec:acronyms}
\input{acronyms.tex}
% If you want glossary uncomment below -- comment out the two lines above
%\printglossaries





\end{document}
